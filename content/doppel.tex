\begin{table}[!htp]
\centering
\caption{Verschiebung vom Nullpunkt und zugehörige Intensitäten vom Doppelspalt.}
\label{tab:doppel}
\begin{tabular}{c c c c}
\toprule
{{Position / mm}} & {{Ausschlag / $\mu$ A}} & {{Position / mm}} & {{Ausschlag / $\mu$ A}} \\
\midrule
-11,00  &  0,0070  &0,50  &  0,4000    \\
-10,50  &  0,0120  &1,00  &  0,1450  \\
-10,00  &  0,0190  &1,50  &  0,0480  \\
-9,50   &  0,0250  &2,00  &  0,1050  \\
-9,00   &  0,0150  &2,50  &  0,1300   \\
-8,50   &  0,0100  &3,00  &  0,0820  \\
-8,25   &  0,0095  &3,50  &  0,0340  \\
-8,00   &  0,0100  &4,00  &  0,0145 \\
-7,50   &  0,0120  &5,00  &  0,0180  \\
-7,00   &  0,0160  &5,50  &  0,0240  \\
-6,75   &  0,0175  &6,00  &  0,0225 \\
-6,50   &  0,0140  &6,50  &  0,0150  \\
-6,00   &  0,0215  &7,00  &  0,0135 \\
-5,50   &  0,0260  &7,50  &  0,1650  \\
-5,00   &  0,0225  &8,00  &  0,0170  \\
-4,50   &  0,0170  &8,50  &  0,0130  \\
-4,00   &  0,0165  &9,00  &  0,0086 \\
-3,50   &  0,4600  &9,50  &  0,0056 \\
-3,00   &  0,1400  &10,00 &  0,0072 \\
-2,50   &  0,2250  &10,50 &  0,0110  \\
-2,00   &  0,1900  &11,00 &  0,0115 \\
-1,50   &  0,7500  &11,50 &  0,0075 \\
-1,00   &  1,4000  &12,00 &  0,0042 \\
-0,50   &  0,4200  &12,50 &  0,0050  \\
0,00    &  0,5600  &  & \\
\bottomrule
\end{tabular}
\end{table}