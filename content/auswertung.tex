\section{Auswertung}
\label{sec:Auswertung}
Zur Auswertung der Messungen an den verschiedenen Spalten ist zunächst bei nicht aktiviertem Laser ein Dunkelstrom von
$I_{\symup{Dunkel}} = 0.175$ nA an der Diode gemessen worden.
Des Weiteren beträgt der Abstand zwischen Diode und Spalt $L = 1.08$ m. Der verwendete Laser hat eine Wellenlänge von $\lambda = 633$nm.
Im Folgenden werden die Breiten der jeweiligen gemessenen Spalte mittels der Fraunhofernäherung bestimmt. Beim Doppelspalt wird zusätzlich
der Abstand der Spalte voneinander berechnet.
\subsection{Bestimmung der Breite des ersten Einzelspaltes}
    Die Verschiebungen $x$ vom Nullpunkt sowie die zugehörigen Intensitäten $I$, welche die Messung am ersten Spalt ergaben sind in \autoref{tab:spalt1}
    zu sehen. Der benötigte Verschiebungswinkel ergibt sich durch $\Phi = \frac{x -x_0}{L}$. Die Verschiebung $x_0$ gibt dabei die 
    Position des Hauptmaximums an, welches sich in der Messung von $0$ unterscheiden kann. Beim ersten Spalt liegt das Hauptmaximum bei 
    $x_0 = - 0.75$ mm.
    \begin{table}[!htp]
\centering
\caption{Verschiebung vom Nullpunkt und zugehörige Intensitäten.}
\label{tab:spalt1}
\begin{tabular}{c c c c}
\toprule
{{Position / mm}} & {{Ausschlag / nA}} & {{Position / mm}} & {{Ausschlag / nA}} \\
\midrule
-25.00   & 0.32  & 1.25    & 12.00 \\
-24.00   & 0.34  &1.50     & 10.50 \\
-23.00   & 0.34  &1.75     & 9.50 \\
-22.00   & 0.32  &2.00     & 8.00 \\
-21.00   & 0.36  &2.25     & 7.00 \\
-20.00   & 0.44  &2.50     & 6.00 \\
-19.00   & 0.48  &2.75     & 5.00 \\
-18.00   & 0.50  &3.00     & 4.00 \\
-17.00   & 0.48  &3.25     & 3.50 \\
-16.00   & 0.54  &3.50     & 2.50 \\
-15.00   & 0.60  &3.75     & 2.00 \\
-14.00   & 0.62  &4.00     & 1.50 \\
-13.00   & 0.50  &4.50     & 1.00 \\
-12.00   & 0.40  &5.00     & 0.84 \\
-11.00   & 0.54  &5.50     & 0.72 \\
-10.50   & 0.74  &6.00     & 0.68 \\
-10.00   & 0.90  &6.50     & 0.70 \\
-9.50    & 1.15  &7.00     & 0.70 \\
-9.00    & 1.30  &7.50     & 0.66 \\
-8.50    & 1.40  &8.00     & 0.60 \\
-8.00    & 1.35  &8.50     & 0.54 \\
-7.00    & 1.00  &9.00     & 0.48 \\
-6.00    & 1.30  &10.00    & 0.38 \\
-5.00    & 3.40  &11.00    & 0.40 \\
-4.00    & 7.40  &12.00    & 0.41 \\
-3.00    & 12.50 &13.00    & 0.40 \\
-2.00    & 17.50 &13.50    & 0.38 \\
-1.75    & 18.50 &14.00    & 0.36 \\
-1.50    & 19.00 &15.00    & 0.34 \\
-1.25    & 19.00 &16.00    & 0.35 \\
-1.00    & 19.50 &17.00    & 0.36 \\
-0.75    & 19.50 &18.00    & 0.36 \\
-0.50    & 19.50 &19.00    & 0.34 \\
-0.25    & 19.00 &20.00    & 0.32 \\
 0.00    & 17.50 &21.00    & 0.30 \\
 0.25    & 16.50 &22.00    & 0.30 \\
 0.50    & 15.50 &23.00    & 0.30 \\
 0.75    & 14.50 &24.00    & 0.29 \\
 1.00    & 13.00 &25.00    & 0.28 \\









\bottomrule
\end{tabular}
\end{table}
    Zur Bestimmung der Breite des Spaltes wird eine Ausgleichskurve aus den Verschiebungswinkeln und den vom Dunkelstrom bereinigten
    Intensitäten der Form von Gleichung \eqref{eqn:intensitaet-einzel} gebildet. 
    Diese ist in \autoref{fig:einzel_mittel} dargestellt. Mittels Python 3.7.0 werden die Amplitude $A_0$ und die gesuchte Spaltbreite $b$ zu 
    \\ \\ 
    \centerline{$A_0 = (1.175 \pm 0.011) $A,}
    \centerline{$b = (1.171 \pm 0.012) \cdot 10^{-4}$m,}
    \\ \\
    bestimmt. Zu Beachten ist hierbei jedoch, dass in der Ausgleichsrechnung der Wert des Hauptmaximums ($I_{\symup{max}}$)
    nicht beachtet wird, da sonst, wie in Gleichung 
    \eqref{eqn:intensitaet-einzel} zu erkennen, durch $\sin(\Phi = 0)$ geteilt werden würde.
    Dieser Wert ist in der Darstellung der Ausgleichskurve markiert.
    \begin{figure}
        \centering
        \includegraphics{build/einzel_mittel.pdf}
        \caption{Auftragung der Winkel gegen die Intensitäten beim ersten Einzelspalt mit Ausgleichskurve.}
        \label{fig:einzel_mittel}
    \end{figure}
    %Neben der Betrachtung des Einzelspaltes unter der Fraunhoferbeugung, kann dieser auch mit der Fouriertransformierten betrachtet werden.
    %Dabei ist die Intensität proportional zum Betragsquadrat der Fouriertransformierten, das heißt nach Gleichung \eqref{eqn:??} zu
    %\begin{equation}
    %\label{eqn:fourier_quadrat}
    %    \lvert g(\Phi) \rvert^2 = A_0^2 \cdot \biggl (\frac{\lambda}{\pi \sin(\Phi)} \biggr)^2 \cdot \sin^2\biggl(\frac{2 \pi b \sin(\Phi)}{\lambda}\biggr). 
    %\end{equation}
    %Mittels Python 3.7.0 wird wieder eine Ausgleichskurve über die Werte aus \autoref{tab:spalt1}, wie zuvor beschrieben, bestimmt, wobei auch hier der Wert des Hauptmaxmimums 
    %nicht beachtet wird. Die Kurve ist in \autoref{fig:em_fourier} zu sehen. 
    %Die daraus resultiereden Werte für Amplitude und Spaltbreite ergeben sich zu:
    %\\ \\ 
    %    \centerline{$A_0 = (1.175 \pm 0.011) $A,}
    %    \centerline{$b = (1.171 \pm 0.012) \cdot 10^{-4}$m.}
    %\\ \\
    %\begin{figure}
    %    \centering
    %    \includegraphics{build/em_fourier.pdf}
    %    \caption{Intensitäten und zugehörige Winkel mit angepasster Fouriertransformation.}
    %    \label{fig:em_fourier}
    %\end{figure}
%
%
\subsection{Bestimmung der Breite des zweiten Einzelspaltes}
    Die Bestimmung der Breite des zweiten Spaltes erfolgt analog zu der des ersten Spaltes. Zu Erwähnen sei hierbei, dass der zweite Spalt kleiner 
    als der erste ist. Die Verschiebung des Hauptmaxmimums liegt diesmal bei $x_0 = 0$, wobei dieses ebenfalls nicht in der Ausgleichsrechnung
    beachtet wird. Die gemessenen Werte für die Verschiebung vom Nullpunkt und zugehörige Intensitäten sind 
    in \autoref{tab:spalt2} zu sehen. Die Verschiebungswinkel bestimmen sich wie bereits zuvor.
    Es wird darüber mittels Python 3.7.0 eine Ausgleichskurve der Form von Gleichung 
    \eqref{eqn:intensitaet-einzel} bestimmt, welche in \autoref{fig:einzel_klein.pdf} zu sehen ist. Die Amplitude und die Spaltbreite sind 
    \\ \\ 
    \centerline{$A_0 = (1.983 \pm 0.039) $A,}
    \centerline{$b = (4.645\pm 0.106) \cdot 10^{-5}$m.}
    \\ \\
    \begin{table}[!htp]
\centering
\caption{Verschiebung vom Nullpunkt und zugehörige Intensitäten vom zweiten Spalt.}
\label{tab:spalt2}
\begin{tabular}{c c c c}
\toprule
{{Position / mm}} & {{Ausschlag / nA}}  & {{Position / mm}} & {{Ausschlag / nA}}\\
\midrule
-25 & 0,36 &0  & 9,40 \\ 
-24 & 0,38 &0,5& 9,00  \\
-23 & 0,42 & 1 & 8,80  \\
-22 & 0,46 & 2 & 8,00   \\
-21 & 0,52 & 3 & 7,20  \\
-20 & 0,58 & 4 & 6,00  \\
-19 & 0,60 & 5 & 5,00  \\
-18 & 0,62 & 6 & 3,80  \\
-17 & 0,62 & 7 & 3,00  \\
-16 & 0,60 & 8 & 2,10  \\
-15 & 0,62 & 9 & 1,60  \\
-14 & 0,65 &10 & 1,20  \\
-13 & 0,75 &11 & 1,00  \\
-12 & 1,05 &12 & 0,90  \\
-11 & 1,45 &13 & 0,94 \\
-10 & 2,05 &14 & 0,96 \\
-9  & 2,80  &15 & 0,86 \\
-8  & 3,80  &16 & 0,82 \\
-7  & 4,80  &17 & 0,78 \\
-6  & 5,60  &18 & 0,72 \\
-5  & 6,60  &19 & 0,64 \\
-4  & 7,40  &21 & 0,56 \\
-3  & 7,80  &22 & 0,50 \\
-2  & 8,20  &23 & 0,44 \\
-1  & 8,40  &24 & 0,38 \\
-0,5& 8,80  &25 & 0,34 \\
  




\bottomrule
\end{tabular}
\end{table}
    \begin{figure}
        \centering
        \includegraphics{build/einzel_klein.pdf}
        \caption{Auftragung der Winkel gegen die Intensitäten beim zweiten Einzelspalt mit Ausgleichskurve.}
        \label{fig:einzel_klein}
    \end{figure}

\subsection{Bestimmung der Breite und des Spaltabstandes beim Doppelspalt}   
    \begin{figure}
        \centering
        \includegraphics{build/doppel.pdf}
        \caption{Auftragung der Winkel gegen die Intensitäten beim Doppelspalt mit Ausgleichskurve.}
        \label{fig:doppel}
    \end{figure}