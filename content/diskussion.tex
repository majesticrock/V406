\section{Diskussion}
\label{sec:Diskussion}
Bei diesem Versuch existieren Fehlerquellen verschiedenster Art. Der Einfluss anderer Lichtquellen, wie der Lampe die benötigt wird um 
die Messwerte vom Amperemeter abzulesen, ist durch die Beachtung des Offsetstroms als gering einzustufen, wobei Schwankungen auftreten können.
Diese treten zum Beispiel durch Anstoßen am Tisch oder eigene Bewegungen beim Messen auf. Durch Anstoßen des Tisches kann sich auch die Blende 
vor der Lichtquelle ein wenig in der Halterung verschieben, was die Messung verfälschen kann. Ein weiterer Faktor ist die begrenzte Genauigkeit
des Amperemeters; Notwendiges Ändern der Messskala führt dazu, dass einige Werte genauer gemessen werden können als andere. Hier ist auch die 
begrenzte Genauigkeit erwähnt, mit der das menschliche Auge von der analogen Skala ablesen kann. 

Die Messung der Einzelspalte ist im Allgemeinen als gut zu bezeichnen, wenngleich die Messung des zweiten Spaltes keine Nebenmaxima aufzeigte.
Das ist auch am, im Vergleich zum ersten Spalt, größeren Fehler zu erkennen. Die Ergebnisse
\begin{align*}
    b_1 &= (1,17 \pm 0,01) \cdot 10^{-4} \symup{m}, \\
    b_2 &= (4,6\pm 0,1) \cdot 10^{-5} \symup{m},
\end{align*}
weisen aber beide einen Wert in der richtigen Größenordnung auf. Verglichen mit den Angaben an den Blenden ergibt sich eine Abweichung von 
$22 \%$ für den ersten Spalt und $38,67 \%$ für den zweiten Spalt von den Theoriewerten
\begin{align*}
     b_1 &= 1,5 \cdot 10^{-4} \symup{m}, \\
     b_2 &= 7,5 \cdot 10^{-5} \symup{m},
\end{align*}
was noch einmal verdeutlicht, dass die Messung des ersten Spaltes besser verlief, als die des zweietn.
Die Abweichungen der Breiten sind aber angesichts der Fehlerquellen gering und die Messung der Einzelspalte daher allgemein als gut einzuschätzen.

Die Auswertung des Doppelspaltes ergab eine Spaltbreite $b_{\symup{D}}$ beziehungsweise Spaltenabstand $s_{\symup{D}}$ von:
\begin{align*}
    s_{\symup{D}} = (4,19 \pm 0,02) \cdot 10^{-3} \symup{m}, \\
    b_{\symup{D}} = (2,62 \pm 0,02) \cdot 10^{-3} \symup{m}.
\end{align*}
An der Apparatur sind Angaben zu den Ausmaßen des Doppelspaltes abzulesen von
 \begin{align*}
        s_{\symup{Theorie}} = 2,5 \cdot 10^{-3} \symup{m}, \\
        b_{\symup{Theorie}} = 1,5 \cdot 10^{-3} \symup{m},
 \end{align*}
womit sich eine relative Abweichung des gemessenen Abstandes von $67,76 \%$ und der gemessenen Breite von $75,47\%$ vom jeweiligen Theoriewert
ergibt. Diese Abweichungen sind in \autoref{fig:doppel} gut veranschaulicht. Sie sind aber im Rahmen der Messgenauigkeit bei diesem Versuch
vertretbar. Die Theoriekurve scheint dort wesentlich breitere
Amplituden als die Kurve aus den Messwerten aufzuweisen, was letztendlich zu einer kompletten Verschiebung dieser zwischen beiden Kurven führt.
Angesichts der großen Fehlerquellen ist dies aber zu Erwarten gewesen. In \autoref{fig:doppel_einzel} ist der Zusammenhang zwischen Einzel- und 
Doppelspalt gut zu erkennen. Die Kurve des Einzelspaltes entspricht, wie in der Abbildung zu sehen, der Einhüllenden der Kurve des Doppelspaltes 
gleicher Breite, womit der Zusammenhang bestätigt ist.