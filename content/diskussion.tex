\section{Diskussion}
\label{sec:Diskussion}
Bei diesem Versuch existieren Fehlerquellen verschiedenster Art. Der Einfluss anderer Lichtquellen, wie der Lampe die benötigt wird um 
die Messwerte vom Amperemeter abzulesen, ist durch die beachtung des Dunkelstroms als gering einzustufen, wobei Schwankungen auftreten können.
Diese treten zum Beispiel durch Anstoßen am Tisch oder eigene Bewegungen beim Messen auf. Durch Anstoßen des Tisches kann sich auch die Blende 
vor der Lichtquelle ein wenig in der Halterung verschieben, was die Messung verfälschen kann. Ein weiterer Faktor ist die begrenzte Genauigkeit
des Amperemeters; Notwendiges Ändern der Messskala führt dazu, dass einige Werte genauer gemessen werden können als andere. Hier ist auch die 
begrenzte Genauigkeit erwähnt, mit der das menschliche Auge von der analogen Skala ablesen kann. 

Die Messung der Einzelspalte ist im Allgemeinen als gut zu bezeichnen, wenngleich die Messung des zweiten Spaltes keine Nebenmaxima aufzeigte.
Das ist auch im Vergleich zum ersten Spalt größeren Fehler zu sehen. Die Ergebnisse
\\ \\
 \centerline{$b_1 = (1.171 \pm 0.012) \cdot 10^{-4}$m,}
 \centerline{$b_2 = (4.645\pm 0.106) \cdot 10^{-5}$m,}
\\ \\
weisen aber beide einen Wert in der richtigen Größenordnung auf; Vergleichswerte liegen bei diesen Spalten nicht vor.
Die Abweichungen der Breiten liegen aber im Rahmen der Messgenauigkeit und sind daher als gut einzuschätzen.
